\documentclass[10pt,]{article}
\usepackage[margin=0.85in]{geometry}
\usepackage[hyphens]{url}
\newcommand*{\authorfont}{\fontfamily{phv}\selectfont}
\usepackage[]{mathpazo}
\usepackage{abstract}
\renewcommand{\abstractname}{}    % clear the title
\renewcommand{\absnamepos}{empty} % originally center
\newcommand{\blankline}{\quad\pagebreak[2]}

\providecommand{\tightlist}{%
  \setlength{\itemsep}{0pt}\setlength{\parskip}{0pt}}
\usepackage{longtable,booktabs}

\usepackage{parskip}
\usepackage{titlesec}
\titlespacing\section{0pt}{12pt plus 4pt minus 2pt}{6pt plus 2pt minus 2pt}
\titlespacing\subsection{0pt}{12pt plus 4pt minus 2pt}{6pt plus 2pt minus 2pt}

\titleformat*{\subsubsection}{\normalsize\itshape}

\usepackage{titling}
\setlength{\droptitle}{-.25cm}

%\setlength{\parindent}{0pt}
%\setlength{\parskip}{6pt plus 2pt minus 1pt}
%\setlength{\emergencystretch}{3em}  % prevent overfull lines

\usepackage[T1]{fontenc}
\usepackage[utf8]{inputenc}

\usepackage{fancyhdr}
\pagestyle{fancy}
\usepackage{lastpage}
\renewcommand{\headrulewidth}{0.3pt}
\renewcommand{\footrulewidth}{0.0pt}
\lhead{}
\chead{}
\rhead{\footnotesize INR 5000: International Relations Core Seminar \textbar{} Reading List -- Fall 2017}
\lfoot{}
\cfoot{\small \thepage/\pageref*{LastPage}}
\rfoot{}

\fancypagestyle{firststyle}
{
\renewcommand{\headrulewidth}{0pt}%
   \fancyhf{}
   \fancyfoot[C]{\small \thepage/\pageref*{LastPage}}
}

%\def\labelitemi{--}
%\usepackage{enumitem}
%\setitemize[0]{leftmargin=25pt}
%\setenumerate[0]{leftmargin=25pt}




\makeatletter
\@ifpackageloaded{hyperref}{}{%
\ifxetex
  \usepackage[setpagesize=false, % page size defined by xetex
              unicode=false, % unicode breaks when used with xetex
              xetex]{hyperref}
\else
  \usepackage[unicode=true]{hyperref}
\fi
}
\@ifpackageloaded{xcolor}{
    \PassOptionsToPackage{usenames,dvipsnames}{xcolor}
}{%
    \usepackage[usenames,dvipsnames]{xcolor}
}
\makeatother
\hypersetup{breaklinks=true,
            bookmarks=true,
            pdfauthor={ ()},
             pdfkeywords = {},
            pdftitle={INR 5000: International Relations Core Seminar \textbar{} Reading List},
            colorlinks=true,
            citecolor=Aquamarine,
            urlcolor=Aquamarine,
            linkcolor=magenta,
            pdfborder={0 0 0}}
\urlstyle{same}  % don't use monospace font for urls


\setcounter{secnumdepth}{0}





\usepackage{setspace}

\title{INR 5000: International Relations Core Seminar \textbar{} Reading List}
\author{Lecturer: Joshua Scriven}
\date{Fall 2017}


\begin{document}

		\maketitle
	

		\thispagestyle{firststyle}

%	\thispagestyle{empty}


	\noindent \begin{tabular*}{\textwidth}{ @{\extracolsep{\fill}} lr @{\extracolsep{\fill}}}


E-mail: \texttt{\href{mailto:jscriven@fsu.edu}{\nolinkurl{jscriven@fsu.edu}}} & Web: \href{http://github.com/joshuascriven}{\tt github.com/joshuascriven}\\
Office Hours: W 09:00-11:30  &  Class Hours: TR 11:45-14:30 p.m.\\
Office: 233 Bellamy Building  & Class Room: \emph{133}\\
	&  \\
	\hline
	\end{tabular*}

\vspace{2mm}



\section{Required Readings}\label{required-readings}

\Urlmuskip=0mu plus 2mu

\subsection{Week 01, 08/15 - 08/19: What is International Relations? How
is it Evaluated? \textbar{} 6
Readings}\label{week-01-0815---0819-what-is-international-relations-how-is-it-evaluated-6-readings}

Dessler, David (1991). ``Beyond Correlations: Toward a Causal Theory of
War''. In: \emph{International Studies Quarterly} 35.3, pp.~337-355

Fearon, James D. (1991). ``Counterfactuals and Hypothesis Testing in
Political Science''. In: \emph{World Politics} 43.02, pp.~169--195.
\url{https://doi.org/10.2307\%2F2010470}.

King, Gary, Robert O. Keohane, and Sidney Verba (1994).
\emph{Designing Social Inquiry: Scientific Inference in Qualitative Research}.
Princeton University Press. Chap. 2. ISBN: 1400821215.

McDermott, Rose (2011). ``New Directions for Experimental Work in
International Relations''. In: \emph{International Studies Quarterly}
55.2, pp.~503-520

Singer, J. David (1961). ``The Level-of-Analysis Problem in
International Relations''. In: \emph{World Politics} 14.1, pp.~77--92.
\url{https://doi.org/10.2307\%2F2009557}.

Waltz, Kenneth N. (1979). ``Laws and Theories''. In:
\emph{Theory of International Politics}. Ed. by Kenneth N. Waltz.
Waveland Press, p.~Chapter 1. ISBN: 1478610530.

\subsubsection{Supplementary \textbar{} 5
Readings:}\label{supplementary-5-readings}

Bueno De Mesquita, Bruce (1985). ``Toward a Scientific Understanding of
International Conflict: A Personal View''. In:
\emph{International Studies Quarterly} 29.2, p.~121.
\url{https://doi.org/10.2307\%2F2600500}.

Jervis, Robert (1985). ``Pluralistic Rigor: A Comment on Bueno De
Mesquita''. In: \emph{International Studies Quarterly} 29.2, p.~145.
\url{https://doi.org/10.2307\%2F2600502}.

McFarland, David D, Charles A. Lave, and James G. March (1977). ``An
Introduction to Models in the Social Sciences.'' In:
\emph{Contemporary Sociology} 6.2, p.~196.
\url{https://doi.org/10.2307\%2F2065803}.

Mearsheimer, J. J. and S. M. Walt (2013). ``Leaving Theory Behind: Why
Simplistic Hypothesis Testing Is Bad for International Relations''. In:
\emph{European Journal of International Relations} 19.3, pp.~427--457.
\url{https://doi.org/10.1177\%2F1354066113494320}.

Walt, Stephen M. (1998). ``International Relations: One World, Many
Theories''. In: \emph{Foreign Policy}, p.~29.
\url{https://doi.org/10.2307\%2F1149275}.

\subsection{Week 02, 08/22 - 08/26: Anarchy and the Paradigms \textbar{}
6
Readings}\label{week-02-0822---0826-anarchy-and-the-paradigms-6-readings}

Keohane, Robert O. (1984).
\emph{After Hegemony: Cooperation and Discord in the World Political Economy}.
Princeton University Press. ISBN: 140082026X.

Mearsheimer, John (2001). ``Anarchy and the Struggle for Power''. In:
\emph{The tragedy of great power politics}. Ed. by John Mearsheimer. WW
Norton \& Company. Chap. 2.

Mearsheimer, John J. (1994). ``The False Promise of International
Institutions''. In: \emph{International Security} 19.3, p.~5.
\url{https://doi.org/10.2307\%2F2539078}.

Moravcsik, Andrew (1997). ``Taking Preferences Seriously: A Liberal
Theory of International Politics''. In:
\emph{International Organization} 51.4, pp.~513--553.
\url{https://doi.org/10.1162\%2F002081897550447}.

Waltz, Kenneth N. (1979). ``Anarchic Structures and Balances of Power''.
In: \emph{Theory of International Politics}. Ed. by Kenneth N. Waltz.
Waveland Press. Chap. 6. ISBN: 1478610530.

Wendt, Alexander (1992). ``Anarchy Is What States Make of It: The Social
Construction of Power Politics''. In: \emph{International Organization}
46.02, p.~391. \url{https://doi.org/10.1017\%2Fs0020818300027764}.

\subsubsection{Supplementary \textbar{} 5
Readings:}\label{supplementary-5-readings-1}

Grieco, Joseph M. (1988). ``Anarchy and the Limits of Cooperation: A
Realist Critique of the Newest Liberal Institutionalism''. In:
\emph{International Organization} 42.03, p.~485.
\url{https://doi.org/10.1017\%2Fs0020818300027715}.

Grieco, Joseph, Robert Powell, and Duncan Snidal (1993). ``The
Relative-Gains Problem for International Cooperation.'' In:
\emph{American Political Science Review} 87.03, pp.~729-743

Hopf, Ted (1998). ``The Promise of Constructivism in International
Relations Theory''. In: \emph{International Security} 23.1, p.~171.
\url{https://doi.org/10.2307\%2F2539267}.

Rathbun, Brian C. (2011). ``Before Hegemony: Generalized Trust and the
Creation and Design of International Security Organizations''. In:
\emph{International Organization} 65.02, pp.~243--273.
\url{https://doi.org/10.1017\%2Fs0020818311000014}.

Ruggie, John Gerard (1998). ``What Makes the World Hang Together?
Neo-Utilitarianism and the Social Constructivist Challenge''. In:
\emph{International Organization} 52.4, pp.~855--885.
\url{https://doi.org/10.1162\%2F002081898550770}.

\subsection{Week 03, 08/29 - 09/02: The State \textbar{} 6
Readings}\label{week-03-0829---0902-the-state-6-readings}

Branch, Jordan (2011). ``Mapping the Sovereign State: Technology,
Authority, and Systemic Change''. In: \emph{International Organization}
65.01, pp.~1--36. \url{https://doi.org/10.1017\%2Fs0020818310000299}.

Hobbes, T. and C.B. Macpherson (1985). \emph{Leviathan}. Pelican books.
Penguin Books Limited. Chap. 17. ISBN: 9780140431957.
\url{https://books.google.com/books?id=n0gm0eaGHQsC}.

Krasner, Stephen D. (1976). ``State Power and the Structure of
International Trade''. In: \emph{World Politics} 28.03, pp.~317--347.
\url{https://doi.org/10.2307\%2F2009974}.

Spruyt, Hendrik (1996). ``The Victory of the Sovereign State''. In:
\emph{The Sovereign State and Its Competitors: An Analysis of Systems Change}.
Ed. by Hendrik Spruyt. Princeton University Press. Chap. 8. ISBN:
0691029105.

Tilly, Charles (1993). ``How War Made States, and Vice Versa''. In:
\emph{Coercion, Capital and European States: Ad 990 - 1992}. Ed. by
Charles Tilly. Wiley-Blackwell. Chap. 3. ISBN: 1557863687.

Weber, Max and E. Matthews (1968). ``Politics as a Vocation''. In:
\emph{Max Weber: Selections in Translation}. Ed. by W. G. Runciman.
Cambridge University Press (CUP), pp.~212--225.
\url{https://doi.org/10.1017\%2Fcbo9780511810831.018}.

\subsubsection{Supplementary \textbar{} 6
Readings:}\label{supplementary-6-readings}

Acemoglu, Daron, Simon Johnson, and James a Robinson (2001). ``The
Colonial Origins of Comparative Development: An Empirical
Investigation''. In: \emph{American Economic Review} 91.5,
pp.~1369--1401. \url{https://doi.org/10.1257\%2Faer.91.5.1369}.

Bean, Richard (1973). ``War and the Birth of the Nation State''. In:
\emph{The Journal of Economic History} 33.01, pp.~203--221.
\url{https://doi.org/10.1017\%2Fs0022050700076531}.

Lake, David a. (1996). ``Anarchy, Hierarchy, and the Variety of
International Relations''. In: \emph{International Organization} 50.01,
p.~1. \url{https://doi.org/10.1017\%2Fs002081830000165x}.

-------- (2003). ``The New Sovereignty in International Relations''. In:
\emph{International Studies Review} 5.3, pp.~303--323.
\url{https://doi.org/10.1046\%2Fj.1079-1760.2003.00503001.x}.

Stasavage, David (2014). ``Was Weber Right? the Role of Urban Autonomy
in Europe's Rise''. In: \emph{American Political Science Review} 108.02,
pp.~337--354. \url{https://doi.org/10.1017\%2Fs0003055414000173}.

Wendt, A. and R. Duvall (2008). ``Sovereignty and the Ufo''. In:
\emph{Political Theory} 36.4, pp.~607--633.
\url{https://doi.org/10.1177\%2F0090591708317902}.

\subsection{Week 04, 09/05 - 09/09: Domestic Politics and Institutions
\textbar{} 6
Readings}\label{week-04-0905---0909-domestic-politics-and-institutions-6-readings}

Bueno De Mesquita, Bruce, Alastair Smith, Randolph M. Siverson, and
James D. Morrow (2003). \emph{The Logic of Political Survival}. MIT
Press. ISBN: 0262261774.

Croco, Sarah E (2011). ``The Decider's Dilemma: Leader Culpability, War
Outcomes, and Domestic Punishment''. In:
\emph{American Political Science Review} 105.03, pp.~457-477

Milner, Helen V. (1997).
\emph{Interests, Institutions, and Information: Domestic Politics and International Relations}.
Princeton University Press. ISBN: 0691011761.

Putnam, Robert D. (1988). ``Diplomacy and Domestic Politics: The Logic
of Two-Level Games''. In: \emph{International Organization} 42.03,
p.~427. \url{https://doi.org/10.1017\%2Fs0020818300027697}.

Rogowski, Ronald (1987). ``Political Cleavages and Changing Exposure to
Trade''. In: \emph{The American Political Science Review} 81.4, p.~1121.
\url{https://doi.org/10.2307\%2F1962581}.

Tomz, Michael (2007). ``Domestic Audience Costs in International
Relations: An Experimental Approach''. In:
\emph{International Organization} 61.04.
\url{https://doi.org/10.1017\%2Fs0020818307070282}.

\subsubsection{Supplementary \textbar{} 6
Readings:}\label{supplementary-6-readings-1}

Clark, David H. (2001). ``Trading Butter for Guns: Domestic Imperatives
for Foreign Policy Substitution''. In:
\emph{Journal of Conflict Resolution} 45.5, pp.~636--660.
\url{https://doi.org/10.1177\%2F0022002701045005005}.

Fearon, James D. (1998). ``DOMESTIC POLITICS, FOREIGN POLICY, AND
THEORIES OF INTERNATIONAL RELATIONS''. In:
\emph{Annual Review of Political Science} 1.1, pp.~289--313.
\url{https://doi.org/10.1146\%2Fannurev.polisci.1.1.289}.

Gourevitch, Peter (1978). ``The Second Image Reversed: The International
Sources of Domestic Politics''. In: \emph{International Organization}
32.04, pp.~881-912

Schultz, Kenneth a. (2005). ``The Politics of Risking Peace: Do Hawks or
Doves Deliver the Olive Branch?'' In: \emph{International Organization}
59.01. \url{https://doi.org/10.1017\%2Fs0020818305050071}.

Tomz, Michael R. and Jessica L. P. Weeks (2013). ``Public Opinion and
the Democratic Peace''. In: \emph{American Political Science Review}
107.04, pp.~849--865.
\url{https://doi.org/10.1017\%2Fs0003055413000488}.

Weeks, Jessica L. (2012). ``Strongmen and Straw Men: Authoritarian
Regimes and the Initiation of International Conflict''. In:
\emph{American Political Science Review} 106.02, pp.~326--347.
\url{https://doi.org/10.1017\%2Fs0003055412000111}.

\subsection{Week 05, 09/12 - 09/16: Transnational and Non-State Actors
\textbar{} 6
Readings}\label{week-05-0912---0916-transnational-and-non-state-actors-6-readings}

Büthe, Tim, Solomon Major, and André De Mello E Souza (2012). ``The
Politics of Private Foreign Aid: Humanitarian Principles, Economic
Development Objectives, and Organizational Interests in Ngo Private Aid
Allocation''. In: \emph{International Organization} 66.04, pp.~571--607.
\url{https://doi.org/10.1017\%2Fs0020818312000252}.

Cooley, Alexander and James Ron (2002). ``The Ngo Scramble:
Organizational Insecurity and the Political Economy of Transnational
Action''. In: \emph{International Security} 27.1, pp.~5--39.
\url{https://doi.org/10.1162\%2F016228802320231217}.

Evangelista, Matthew (1995). ``The Paradox of State Strength:
Transnational Relations, Domestic Structures, and Security Policy in
Russia and the Soviet Union''. In: \emph{International Organization}
49.01, p.~1. \url{https://doi.org/10.1017\%2Fs0020818300001569}.

Horowitz, Michael C. (2010). ``Nonstate Actors and the Diffusion of
Innovations: The Case of Suicide Terrorism''. In:
\emph{International Organization} 64.01, p.~33.
\url{https://doi.org/10.1017\%2Fs0020818309990233}.

Keck, Margaret E. and Kathryn Sikkink (1999). ``Transnational Advocacy
Networks in International and Regional Politics''. In:
\emph{International Social Science Journal} 51.159, pp.~89--101.
\url{https://doi.org/10.1111\%2F1468-2451.00179}.

Moravcsik, Andrew (1991). ``Negotiating the Single European Act:
National Interests and Conventional Statecraft in the European
Community''. In: \emph{International Organization} 45.01, p.~19.
\url{https://doi.org/10.1017\%2Fs0020818300001387}.

\subsubsection{Supplementary \textbar{} 7
Readings:}\label{supplementary-7-readings}

Büthe, Tim (2004). ``Governance Through Private Authority? Non-State
Actors in World Politics''. In: \emph{Journal of International Affairs}
58.1, pp.~281--290.

Finnemore, Martha and Kathryn Sikkink (1998). ``International Norm
Dynamics and Political Change''. In: \emph{International Organization}
52.4, pp.~887--917. \url{https://doi.org/10.1162\%2F002081898550789}.

Gilli, Andrea and Mauro Gilli (2014). ``The Spread of Military
Innovations: Adoption Capacity Theory, Tactical Incentives, and the Case
of Suicide Terrorism''. In: \emph{Security Studies} 23.3, pp.~513--547.
\url{https://doi.org/10.1080\%2F09636412.2014.935233}.

Hyde, Susan D. (2011). ``Catch Us if You Can: Election Monitoring and
International Norm Diffusion''. In:
\emph{American Journal of Political Science} 55.2, pp.~356--369.
\url{https://doi.org/10.1111\%2Fj.1540-5907.2011.00508.x}.

Kim, Dongwook (2013). ``International Nongovernmental Organizations and
the Global Diffusion of National Human Rights Institutions''. In:
\emph{International Organization} 67.03, pp.~505--539.
\url{https://doi.org/10.1017\%2Fs0020818313000131}.

Murdie, Amanda M. and David R. Davis (2011). ``Shaming and Blaming:
Using Events Data to Assess the Impact of Human Rights INGOs1''. In:
\emph{International Studies Quarterly} 56.1, pp.~1--16.
\url{https://doi.org/10.1111\%2Fj.1468-2478.2011.00694.x}.

Tallberg, Jonas, Thomas Sommerer, Theresa Squatrito, and Christer
Jönsson (2014). ``Explaining the Transnational Design of International
Organizations''. In: \emph{International Organization} 68.04,
pp.~741--774. \url{https://doi.org/10.1017\%2Fs0020818314000149}.

\subsection{Week 06, 09/19 - 09/23: Strategic Choice Approach \textbar{}
6
Readings}\label{week-06-0919---0923-strategic-choice-approach-6-readings}

Frieden, Jeffry a. (1999). ``Actors and Preferences in International
Relations''. In: \emph{Strategic Choice and International Relations}.
Ed. by David A. Lake and Robert Powell. Princeton University Press.
Chap. 2, pp.~39-76. ISBN: 0691026971.

Jervis, Robert (1988). ``War and Misperception''. In:
\emph{Journal of Interdisciplinary History} 18.4, p.~675.
\url{https://doi.org/10.2307\%2F204820}.

Lake, David a and Robert Powell (1999). ``International Relations: A
Strategic-Choice Approach''. In:
\emph{Strategic choice and international relations}. Ed. by David A Lake
and Robert Powell. Princeton University Press Princeton, NJ. Chap. 1.

Morrow, James D (1999). ``The Strategic Setting of Choices: Signaling,
Commitment, and Negotiation in International Politics''. In:
\emph{Strategic choice and international relations}. Ed. by David A Lake
and Robert Powell. Princeton University Press Princeton, NJ, pp.~77-114.

Rogowski, Ronald (1999). ``Institutions as Constraints on Strategic
Choice''. In: \emph{Strategic choice and international relations}. Ed.
by David A Lake and Robert Powell. Princeton: Princeton University
Press, pp.~115-136.

Walt, Stephen M. (1999). ``Rigor or Rigor Mortis? Rational Choice and
Security Studies''. In: \emph{International Security} 23.4, pp.~5--48.
\url{https://doi.org/10.1162\%2Fisec.23.4.5}.

\subsubsection{Supplementary \textbar{} 4
Readings:}\label{supplementary-4-readings}

Herrmann, Richard K. and Michael P. Fischerkeller (1995). ``Beyond the
Enemy Image and Spiral Model: Cognitive-Strategic Research After the
Cold War''. In: \emph{International Organization} 49.03, p.~415.
\url{https://doi.org/10.1017\%2Fs0020818300033336}.

Levy, Jack S. (1997). ``Prospect Theory, Rational Choice, and
International Relations''. In: \emph{International Studies Quarterly}
41.1, pp.~87--112. \url{https://doi.org/10.1111\%2F0020-8833.00034}.

Mercer, Jonathan (2005). ``Rationality and Psychology in International
Politics''. In: \emph{International Organization} 59.01.
\url{https://doi.org/10.1017\%2Fs0020818305050058}.

Zagare, Frank C. (1999). ``All Mortis, No Rigor''. In:
\emph{International Security} 24.2, pp.~107--114.
\url{https://doi.org/10.1162\%2F016228899560185}.

\subsection{Week 07, 09/26 - 09/30: Interstate Conflict \textbar{} 6
Readings}\label{week-07-0926---0930-interstate-conflict-6-readings}

Fearon, James D (1994). ``Domestic Political Audiences and the
Escalation of International Disputes.'' In:
\emph{American Political Science Review} 88.03, pp.~577-592

-------- (1995). ``Rationalist Explanations for War''. In:
\emph{International organization} 49.03, pp.~379-414

Jervis, Robert (1978). ``Cooperation Under the Security Dilemma''. In:
\emph{World politics} 30.02, pp.~167-214

Kenwick, Michael R, John a Vasquez, and Matthew A Powers (2015). ``Do
Alliances Really Deter?'' In: \emph{The Journal of Politics} 77.4,
pp.~943-954

Powell, Robert (2006). ``War as a Commitment Problem''. In:
\emph{International organization} 60.01, pp.~169-203

Schultz, Kenneth A (1999). ``Do Democratic Institutions Constrain or
Inform? Contrasting Two Institutional Perspectives on Democracy and
War''. In: \emph{International Organization} 53.02, pp.~233-266

\subsubsection{Supplementary \textbar{} 4
Readings:}\label{supplementary-4-readings-1}

Benson, Brett V. (2011). ``Unpacking Alliances: Deterrent and Compellent
Alliances and Their Relationship With Conflict, 1816-2000''. In:
\emph{The Journal of Politics} 73.4, pp.~1111--1127.
\url{https://doi.org/10.1017\%2Fs0022381611000867}.

Johnson, Jesse C and Brett Ashley Leeds (2011). ``Defense Pacts: A
Prescription for Peace? 1''. In: \emph{Foreign Policy Analysis} 7.1,
pp.~45-65

Smith, Alastair (1995). ``Alliance Formation and War''. In:
\emph{International Studies Quarterly} 39.4, pp.~405-425

Wagner, R Harrison (2000). ``Bargaining and War''. In:
\emph{American Journal of Political Science}, pp.~469-484

\subsection{Week 08, 10/03 - 10/07: Civil Conflict \textbar{} 6
Readings}\label{week-08-1003---1007-civil-conflict-6-readings}

Fearon, James D and David D Laitin (2003). ``Ethnicity, Insurgency, and
Civil War''. In: \emph{American political science review} 97.01,
pp.~75-90

Kalyvas, Stathis N and Laia Balcells (2010). ``International System and
Technologies of Rebellion: How the End of the Cold War Shaped Internal
Conflict''. In: \emph{American Political Science Review} 104.03,
pp.~415-429

Leeds, Brett Ashley, Michaela Mattes, and Jeremy S. Vogel (2009).
``Interests, Institutions, and the Reliability of International
Commitments''. In: \emph{American Journal of Political Science} 53.2,
pp.~461--476. \url{https://doi.org/10.1111\%2Fj.1540-5907.2009.00381.x}.

Salehyan, Idean (2008). ``The Externalities of Civil Strife: Refugees as
a Source of International Conflict''. In:
\emph{American Journal of Political Science} 52.4, pp.~787-801

Walter, Barbara F (2009). ``Bargaining Failures and Civil War''. In:
\emph{Annual Review of Political Science} 12, pp.~243-261

Wood, Reed M, Jacob D Kathman, and Stephen E Gent (2012). ``Armed
Intervention and Civilian Victimization in Intrastate Conflicts''. In:
\emph{Journal of Peace Research} 49.5, pp.~647-660

\subsubsection{Supplementary \textbar{} 5
Readings:}\label{supplementary-5-readings-2}

Fjelde, H. and D. Nilsson (2012). ``Rebels Against Rebels: Explaining
Violence Between Rebel Groups''. In:
\emph{Journal of Conflict Resolution} 56.4, pp.~604--628.
\url{https://doi.org/10.1177\%2F0022002712439496}.

Leeds, Brett Ashley (2003). ``Alliance Reliability in Times of War:
Explaining State Decisions to Violate Treaties''. In:
\emph{International Organization} 57.04, pp.~801-827

Pearlman, W. and K. G. Cunningham (2011). ``Nonstate Actors,
Fragmentation, and Conflict Processes''. In:
\emph{Journal of Conflict Resolution} 56.1, pp.~3--15.
\url{https://doi.org/10.1177\%2F0022002711429669}.

Salehyan, Idean and Kristian Skrede Gleditsch (2006). ``Refugees and the
Spread of Civil War''. In: \emph{International Organization},
pp.~335-366

Salehyan, Idean, Kristian Skrede Gleditsch, and David E. Cunningham
(2011). ``Explaining External Support for Insurgent Groups''. In:
\emph{International Organization} 65.04, pp.~709--744.
\url{https://doi.org/10.1017\%2Fs0020818311000233}.

\subsection{Week 09, 10/10 - 10/14: Conflict Duration \& Resolution
\textbar{} 6
Readings}\label{week-09-1010---1014-conflict-duration-resolution-6-readings}

Goemans, Hein E (2000). ``Fighting for Survival the Fate of Leaders and
the Duration of War''. In: \emph{Journal of Conflict Resolution} 44.5,
pp.~555-579

Sawyer, Katherine, Kathleen Gallagher Cunningham, and William Reed
(2015). ``The Role of External Support in Civil War Termination''. In:
\emph{Journal of Conflict Resolution}, p.~002200271560076.
\url{https://doi.org/10.1177\%2F0022002715600761}.

Shannon, Megan, Daniel Morey, and Frederick J. Boehmke (2010). ``The
Influence of International Organizations on Militarized Dispute
Initiation and Duration1''. In: \emph{International Studies Quarterly}
54.4, pp.~1123--1141.
\url{https://doi.org/10.1111\%2Fj.1468-2478.2010.00629.x}.

Slantchev, Branislav L (2004). ``How Initiators End Their Wars: The
Duration of Warfare and the Terms of Peace''. In:
\emph{American Journal of Political Science} 48.4, pp.~813-829

Tiernay, Michael (2015). ``Killing Kony''. In:
\emph{Journal of Conflict Resolution} 59.2, pp.~175--206.
\url{https://doi.org/10.1177\%2F0022002713499720}.

Walter, Barbara F (1997). ``The Critical Barrier to Civil War
Settlement''. In: \emph{International organization} 51.03, pp.~335-364

\subsubsection{Supplementary \textbar{} 6
Readings:}\label{supplementary-6-readings-2}

Filson, Darren and Suzanne Werner (2002). ``A Bargaining Model of War
and Peace: Anticipating the Onset, Duration, and Outcome of War''. In:
\emph{American Journal of Political Science}, pp.~819-837

Findley, M. G. (2012). ``Bargaining and the Interdependent Stages of
Civil War Resolution''. In: \emph{Journal of Conflict Resolution} 57.5,
pp.~905--932. \url{https://doi.org/10.1177\%2F0022002712453703}.

Fortna, Virginia Page (2004). ``Does Peacekeeping Keep Peace?
International Intervention and the Duration of Peace After Civil War''.
In: \emph{International studies quarterly} 48.2, pp.~269-292

Hartzell, Caroline, Matthew Hoddie, and Donald Rothchild (2001).
``Stabilizing the Peace After Civil War: An Investigation of Some Key
Variables''. In: \emph{International Organization} 55.1, pp.~183--208.
\url{https://doi.org/10.1162\%2F002081801551450}.

Svolik, Milan (2006). ``Lies, Defection, and the Pattern of
International Cooperation''. In:
\emph{American Journal of Political Science} 50.4, pp.~909--925.
\url{https://doi.org/10.1111\%2Fj.1540-5907.2006.00223.x}.

Thyne, Clayton L. (2012). ``Information, Commitment, and Intra-War
Bargaining: The Effect of Governmental Constraints on Civil War
Duration1''. In: \emph{International Studies Quarterly} 56.2,
pp.~307--321. \url{https://doi.org/10.1111\%2Fj.1468-2478.2012.00719.x}.

\subsection{Week 10, 10/17 - 10/21: International Institutions: Design
\textbar{} 6
Readings}\label{week-10-1017---1021-international-institutions-design-6-readings}

Abbott, Kenneth W and Duncan Snidal (2000). ``Hard and Soft Law in
International Governance''. In: \emph{International organization} 54.03,
pp.~421-456

Allee, Todd and Clint Peinhardt (2013). ``Evaluating Three Explanations
for the Design of Bilateral Investment Treaties''. In:
\emph{World Politics} 66.01, pp.~47--87.
\url{https://doi.org/10.1017\%2Fs0043887113000324}.

Blake, Daniel J. (2013). ``Thinking Ahead: Government Time Horizons and
the Legalization of International Investment Agreements''. In:
\emph{International Organization} 67.04, pp.~797--827.
\url{https://doi.org/10.1017\%2Fs0020818313000258}.

Gilligan, Michael J (2004). ``Is There a Broader-Deeper Trade-Off in
International Multilateral Agreements?'' In:
\emph{International Organization} 58.03, pp.~459-484

Johnson, Tana and Johannes Urpelainen (2014). ``International
Bureaucrats and the Formation of Intergovernmental Organizations:
Institutional Design Discretion Sweetens the Pot''. In:
\emph{International Organization} 68.01, pp.~177--209.
\url{https://doi.org/10.1017\%2Fs0020818313000349}.

Koremenos, Barbara, Charles Lipson, and Duncan Snidal (2001). ``The
Rational Design of International Institutions''. In:
\emph{International Organization} 55.4, pp.~761--799.
\url{https://doi.org/10.1162\%2F002081801317193592}.

\subsubsection{Supplementary \textbar{} 8
Readings:}\label{supplementary-8-readings}

Abbott, Kenneth W. and Duncan Snidal (1998). ``Why States Act Through
Formal International Organizations''. In:
\emph{Journal of Conflict Resolution} 42.1, pp.~3--32.
\url{https://doi.org/10.1177\%2F0022002798042001001}.

Copelovitch, Mark S. and Tonya L. Putnam (2014). ``Design in Context:
Existing International Agreements and New Cooperation''. In:
\emph{International Organization} 68.02, pp.~471--493.
\url{https://doi.org/10.1017\%2Fs0020818313000441}.

Donno, Daniela, Shawna K Metzger, and Bruce Russett (2015). ``Screening
Out Risk: IGOs, Member State Selection, and Interstate Conflict,
1951--2000''. In: \emph{International Studies Quarterly} 59.2,
pp.~251-263

Hill Jr, Daniel W (2016). ``Avoiding Obligation: Reservations to Human
Rights Treaties''. In: \emph{Journal of Conflict Resolution} 60.6,
pp.~1129-1158

Kahler, Miles (2000). ``Conclusion: The Causes and Consequences of
Legalization''. In: \emph{International Organization} 54.03, pp.~661-683

Koremenos, Barbara (2005). ``Contracting Around International
Uncertainty''. In: \emph{American Political Science Review} 99.04,
pp.~549--565. \url{https://doi.org/10.1017\%2Fs0003055405051877}.

Mattes, Michaela (2012). ``Reputation, Symmetry, and Alliance Design''.
In: \emph{International Organization} 66.04, pp.~679--707.
\url{https://doi.org/10.1017\%2Fs002081831200029x}.

Wendt, Alexander (2001). ``Driving With the Rearview Mirror: On the
Rational Science of Institutional Design''. In:
\emph{International Organization} 55.4, pp.~1019--1049.
\url{https://doi.org/10.1162\%2F002081801317193682}.

\subsection{Week 11, 10/24 - 10/28: International Institutions:
Compliance, Influence, and Enforcement \textbar{} 6
Readings}\label{week-11-1024---1028-international-institutions-compliance-influence-and-enforcement-6-readings}

Busch, Marc L. (2007). ``Overlapping Institutions, Forum Shopping, and
Dispute Settlement in International Trade''. In:
\emph{International Organization} 61.04.
\url{https://doi.org/10.1017\%2Fs0020818307070257}.

Chapman, Terrence L and Stephen Chaudoin (2013). ``Ratification Patterns
and the International Criminal Court1''. In:
\emph{International Studies Quarterly} 57.2, pp.~400-409

Chayes, Abram and Antonia Handler Chayes (1993). ``On Compliance''. In:
\emph{International Organization} 47.02, p.~175.
\url{https://doi.org/10.1017\%2Fs0020818300027910}.

Cole, Wade M (2015). ``Mind the Gap: State Capacity and the
Implementation of Human Rights Treaties''. In:
\emph{International Organization} 69.02, pp.~405-441

Simmons, Beth a and Daniel J Hopkins (2005). ``The Constraining Power of
International Treaties: Theory and Methods''. In:
\emph{American Political Science Review} 99.04, pp.~623-631

Von Stein, Jana (2005). ``Do Treaties Constrain or Screen? Selection
Bias and Treaty Compliance''. In:
\emph{American Political Science Review} 99.04, pp.~611-622

\subsubsection{Supplementary \textbar{} 7
Readings:}\label{supplementary-7-readings-1}

Chilton, Adam and Dustin Tingley (2013). ``Why the Study of
International Law Needs Experiments''. In:
\emph{Colum. J. Transnat'l L.} 52, p.~173.

Donno, Daniela (2010). ``Who Is Punished? Regional Intergovernmental
Organizations and the Enforcement of Democratic Norms''. In:
\emph{International Organization} 64.04, pp.~593--625.
\url{https://doi.org/10.1017\%2Fs0020818310000202}.

Gray, Julia and Jonathan B Slapin (2012). ``How Effective Are
Preferential Trade Agreements? Ask the Experts''. In:
\emph{The Review of International Organizations} 7.3, pp.~309-333

Huth, Paul K, Sarah E. Croco, and Benjamin J. Appel (2011). ``Does
International Law Promote the Peaceful Settlement of International
Disputes? Evidence From the Study of Territorial Conflicts Since 1945''.
In: \emph{American Political Science Review} 105.02, pp.~415--436.
\url{https://doi.org/10.1017\%2Fs0003055411000062}.

Morrow, James D. (2007). ``When Do States Follow the Laws of War?'' In:
\emph{American Political Science Review} 101.03, pp.~559--572.
\url{https://doi.org/10.1017\%2Fs000305540707027x}.

Pelc, Krzysztof J. (2014). ``The Politics of Precedent in International
Law: A Social Network Application''. In:
\emph{American Political Science Review} 108.03, pp.~547--564.
\url{https://doi.org/10.1017\%2Fs0003055414000276}.

Wotipka, Christine Min and Kiyoteru Tsutsui (2008). ``Global Human
Rights and State Sovereignty: State Ratification of International Human
Rights Treaties, 1965-20011''. In: \emph{Sociological Forum} 23.4,
pp.~724--754. \url{https://doi.org/10.1111\%2Fj.1573-7861.2008.00092.x}.

\subsection{Week 12, 10/31 - 11/04: International Institutions: Human
Rights Regime \textbar{} 6
Readings}\label{week-12-1031---1104-international-institutions-human-rights-regime-6-readings}

Chaudoin, Stephen (2016). ``How Contestation Moderates the Effects of
International Institutions: The International Criminal Court and
Kenya''. In: \emph{The Journal of Politics} 78.2, pp.~557--571.
\url{https://doi.org/10.1086\%2F684595}.

Conrad, Courtenay R. and Emily Hencken Ritter (2013). ``Treaties,
Tenure, and Torture: The Conflicting Domestic Effects of International
Law''. In: \emph{The Journal of Politics} 75.2, pp.~397--409.
\url{https://doi.org/10.1017\%2Fs0022381613000091}.

Finnemore, Martha (1996). ``Constructing Norms of Humanitarian
Intervention''. In:
\emph{The culture of national security: Norms and identity in world politics}
153.

Hill, Daniel W. (2010). ``Estimating the Effects of Human Rights
Treaties on State Behavior''. In: \emph{The Journal of Politics} 72.4,
pp.~1161--1174. \url{https://doi.org/10.1017\%2Fs0022381610000599}.

Neumayer, E. (2005). ``Do International Human Rights Treaties Improve
Respect for Human Rights?'' In: \emph{Journal of Conflict Resolution}
49.6, pp.~925--953. \url{https://doi.org/10.1177\%2F0022002705281667}.

Simmons, Beth a. and Allison Danner (2010). ``Credible Commitments and
the International Criminal Court''. In:
\emph{International Organization} 64.02, p.~225.
\url{https://doi.org/10.1017\%2Fs0020818310000044}.

\subsubsection{Supplementary \textbar{} 23
Readings:}\label{supplementary-23-readings}

Caplan, Lee M. (2003). ``State Immunity, Human Rights, and Jus Cogens: A
Critique of the Normative Hierarchy Theory''. In:
\emph{The American Journal of International Law} 97.4, p.~741.
\url{https://doi.org/10.2307\%2F3133679}.

Cingranelli, David and Mikhail Filippov (2010). ``Electoral Rules and
Incentives to Protect Human Rights''. In: \emph{The Journal of Politics}
72.1, pp.~243--257. \url{https://doi.org/10.1017\%2Fs0022381609990594}.

Cohen, J. L. (2008). ``Rethinking Human Rights, Democracy, and
Sovereignty in the Age of Globalization''. In: \emph{Political Theory}
36.4, pp.~578--606. \url{https://doi.org/10.1177\%2F0090591708317901}.

Cole, Wade M. (2005). ``Sovereignty Relinquished? Explaining Commitment
to the International Human Rights Covenants, 1966-1999''. In:
\emph{American Sociological Review} 70.3, pp.~472--495.
\url{https://doi.org/10.1177\%2F000312240507000306}.

Davenport, Christian and David a. Armstrong (2004). ``Democracy and the
Violation of Human Rights: A Statistical Analysis From 1976 to 1996''.
In: \emph{American Journal of Political Science} 48.3, pp.~538--554.
\url{https://doi.org/10.1111\%2Fj.0092-5853.2004.00086.x}.

Donnelly, Jack (1986). ``International Human Rights: A Regime
Analysis''. In: \emph{International Organization} 40.03, p.~599.
\url{https://doi.org/10.1017\%2Fs0020818300027296}.

Fariss, Christopher J. (2014). ``Respect for Human Rights Has Improved
Over Time: Modeling the Changing Standard of Accountability''. In:
\emph{American Political Science Review} 108.02, pp.~297--318.
\url{https://doi.org/10.1017\%2Fs0003055414000070}.

Finnemore, Martha and Kathryn Sikkink (1998). ``International Norm
Dynamics and Political Change''. In: \emph{International Organization}
52.4, pp.~887--917. \url{https://doi.org/10.1162\%2F002081898550789}.

Ghanea, Nazila (2006). ``I. From Un Commission on Human Rights to Un
Human Rights Council: One Step Forwards or Two Steps Sideways?'' In:
\emph{International and Comparative Law Quarterly} 55.03, pp.~695--705.
\url{https://doi.org/10.1093\%2Ficlq\%2Flei112}.

Hafner-Burton, Emilie M. (2005). ``Trading Human Rights: How
Preferential Trade Agreements Influence Government Repression''. In:
\emph{International Organization} 59.03.
\url{https://doi.org/10.1017\%2Fs0020818305050216}.

-------- (2008). ``Sticks and Stones: Naming and Shaming the Human
Rights Enforcement Problem''. In: \emph{International Organization}
62.04, p.~689. \url{https://doi.org/10.1017\%2Fs0020818308080247}.

Hathaway, Oona a. (2007). ``Why Do Countries Commit to Human Rights
Treaties?'' In: \emph{Journal of Conflict Resolution} 51.4,
pp.~588--621. \url{https://doi.org/10.1177\%2F0022002707303046}.

Keith, Linda Camp, C. Neal Tate, and Steven C. Poe (2009). ``Is the Law
a Mere Parchment Barrier to Human Rights Abuse?'' In:
\emph{The Journal of Politics} 71.2, pp.~644--660.
\url{https://doi.org/10.1017\%2Fs0022381609090513}.

Kelley, Judith (2007). ``Who Keeps International Commitments and Why?
the International Criminal Court and Bilateral Nonsurrender
Agreements''. In: \emph{American Political Science Review} 101.03,
pp.~573-589

Kim, Hunjoon and Kathryn Sikkink (2010). ``Explaining the Deterrence
Effect of Human Rights Prosecutions for Transitional Countries1''. In:
\emph{International Studies Quarterly} 54.4, pp.~939--963.
\url{https://doi.org/10.1111\%2Fj.1468-2478.2010.00621.x}.

Krain, Matthew (2012). ``J'accuse! Does Naming and Shaming Perpetrators
Reduce the Severity of Genocides or Politicides?1''. In:
\emph{International Studies Quarterly} 56.3, pp.~574--589.
\url{https://doi.org/10.1111\%2Fj.1468-2478.2012.00732.x}.

Lupu, Yonatan (2013). ``Best Evidence: The Role of Information in
Domestic Judicial Enforcement of International Human Rights
Agreements''. In: \emph{International Organization} 67.03, pp.~469--503.
\url{https://doi.org/10.1017\%2Fs002081831300012x}.

Meernik, James, Rosa Aloisi, Marsha Sowell, and Angela Nichols (2012).
``The Impact of Human Rights Organizations on Naming and Shaming
Campaigns''. In: \emph{Journal of Conflict Resolution} 56.2,
pp.~233--256. \url{https://doi.org/10.1177\%2F0022002711431417}.

Moravcsik, Andrew (2000). ``The Origins of Human Rights Regimes:
Democratic Delegation in Postwar Europe''. In:
\emph{International Organization}, pp.~217-252

Mutua, Makau (2009). ``An Apology for a Pathological Brute''. In:
\emph{Human Rights Quarterly} 31.3, pp.~806--809.
\url{https://doi.org/10.1353\%2Fhrq.0.0089}.

Powell, Emilia Justyna and Jeffrey K. Staton (2009). ``Domestic Judicial
Institutions and Human Rights Treaty Violation''. In:
\emph{International Studies Quarterly} 53.1, pp.~149--174.
\url{https://doi.org/10.1111\%2Fj.1468-2478.2008.01527.x}.

Vreeland, James Raymond (2008). ``Political Institutions and Human
Rights: Why Dictatorships Enter Into the United Nations Convention
Against Torture''. In: \emph{International Organization} 62.01.
\url{https://doi.org/10.1017\%2Fs002081830808003x}.

Wallace, Geoffrey P.R. (2013). ``International Law and Public Attitudes
Toward Torture: An Experimental Study''. In:
\emph{International Organization} 67.01, pp.~105--140.
\url{https://doi.org/10.1017\%2Fs0020818312000343}.

\subsection{Week 13, 11/07 - 11/11: International Political Economy:
Trade and Foreign Direct Investment \textbar{} 6
Readings}\label{week-13-1107---1111-international-political-economy-trade-and-foreign-direct-investment-6-readings}

Bechtel, Michael M and Thomas Sattler (2015). ``What Is Litigation in
the World Trade Organization Worth?'' In:
\emph{International Organization} 69.02, pp.~375-403

Ehrlich, Sean D. (2007). ``Access to Protection: Domestic Institutions
and Trade Policy in Democracies''. In: \emph{International Organization}
61.03. \url{https://doi.org/10.1017\%2Fs0020818307070191}.

Goldstein, Judith L, Douglas Rivers, and Michael Tomz (2007).
``Institutions in International Relations: Understanding the Effects of
the Gatt and the Wto on World Trade''. In:
\emph{International Organization} 61.01.
\url{https://doi.org/10.1017\%2Fs0020818307070014}.

Guisinger, Alexandra (2009). ``Determining Trade Policy: Do Voters Hold
Politicians Accountable?'' In: \emph{International Organization} 63.03,
p.~533. \url{https://doi.org/10.1017\%2Fs0020818309090183}.

Johnson, Jesse C, Mark Souva, and Dale L. Smith (2013).
``Market-Protecting Institutions and the World Trade Organization's
Ability to Promote Trade''. In: \emph{International Studies Quarterly}
57.2, pp.~410--417. \url{https://doi.org/10.1111\%2Fisqu.12077}.

Kono, Daniel Y (2006). ``Optimal Obfuscation: Democracy and Trade Policy
Transparency''. In: \emph{American Political Science Review} 100.03,
pp.~369-384

\subsubsection{Supplementary \textbar{} 28
Readings:}\label{supplementary-28-readings}

Ardanaz, Martin, M. Victoria Murillo, and Pablo M. Pinto (2013).
``Sensitivity to Issue Framing on Trade Policy Preferences: Evidence
From a Survey Experiment''. In: \emph{International Organization} 67.02,
pp.~411--437. \url{https://doi.org/10.1017\%2Fs0020818313000076}.

Bearce, David H, Cody D. Eldredge, and Brandy J. Jolliff (2014). ``Do
Finite Duration Provisions Reduce International Bargaining Delay?'' In:
\emph{International Organization} 69.01, pp.~219--239.
\url{https://doi.org/10.1017\%2Fs0020818314000162}.

Bernhard, William and David Leblang (1999). ``Democratic Institutions
and Exchange-Rate Commitments''. In: \emph{International Organization}
53.01, pp.~71-97

Büthe, Tim and Helen V. Milner (2013). ``Foreign Direct Investment and
Institutional Diversity in Trade Agreements: Credibility, Commitment,
and Economic Flows in the Developing World, 1971-2007''. In:
\emph{World Politics} 66.01, pp.~88--122.
\url{https://doi.org/10.1017\%2Fs0043887113000336}.

Chase, Kerry a. (2003). ``Economic Interests and Regional Trading
Arrangements: The Case of Nafta''. In: \emph{International Organization}
57.01. \url{https://doi.org/10.1017\%2Fs0020818303571053}.

Copelovitch, Mark S. and Jon C.W. Pevehouse (2013). ``Ties That Bind?
Preferential Trade Agreements and Exchange Rate Policy Choice''. In:
\emph{International Studies Quarterly} 57.2, pp.~385--399.
\url{https://doi.org/10.1111\%2Fisqu.12050}.

Davis, Christina L. and Yuki Shirato (2007). ``Firms, Governments, and
Wto Adjudication: Japan's Selection of Wto Disputes''. In:
\emph{World Politics} 59.02, pp.~274--313.
\url{https://doi.org/10.1353\%2Fwp.2007.0021}.

Fordham, Benjamin O and Katja B Kleinberg (2012). ``How Can Economic
Interests Influence Support for Free Trade?'' In:
\emph{International Organization} 66.02, pp.~311-328

Frieden, Jeffry A (1991). ``Invested Interests: The Politics of National
Economic Policies in a World of Global Finance''. In:
\emph{International Organization} 45.04, pp.~425-451

Garrett, Geoffrey (1998). ``Global Markets and National Politics:
Collision Course or Virtuous Circle?'' In:
\emph{International Organization} 52.4, pp.~787--824.
\url{https://doi.org/10.1162\%2F002081898550752}.

Gowa, Joanne and Soo Yeon Kim (2005). ``An Exclusive Country Club: The
Effects of the Gatt on Trade, 1950-94''. In: \emph{World Politics}
57.04, pp.~453--478. \url{https://doi.org/10.1353\%2Fwp.2006.0010}.

Gowa, Joanne and Edward D. Mansfield (1993). ``Power Politics and
International Trade.'' In: \emph{American Political Science Review}
87.02, pp.~408--420. \url{https://doi.org/10.2307\%2F2939050}.

Hafner-Burton, Emilie M. (2005). ``Trading Human Rights: How
Preferential Trade Agreements Influence Government Repression''. In:
\emph{International Organization} 59.03.
\url{https://doi.org/10.1017\%2Fs0020818305050216}.

Hanson, Gordon H, Kenneth Scheve, and Matthew J Slaughter (2007).
``Public Finance and Individual Preferences Over Globalization
Strategies''. In: \emph{Economics \& Politics} 19.1, pp.~1-33

Hiscox, Michael J. (2001). ``Class Versus Industry Cleavages:
Inter-Industry Factor Mobility and the Politics of Trade''. In:
\emph{International Organization} 55.1, pp.~1--46.
\url{https://doi.org/10.1162\%2F002081801551405}.

-------- (2006). ``Through a Glass and Darkly: Attitudes Toward
International Trade and the Curious Effects of Issue Framing''. In:
\emph{International Organization} 60.03.
\url{https://doi.org/10.1017\%2Fs0020818306060255}.

Jensen, Nathan M. (2003). ``Democratic Governance and Multinational
Corporations: Political Regimes and Inflows of Foreign Direct
Investment''. In: \emph{International Organization} 57.03.
\url{https://doi.org/10.1017\%2Fs0020818303573040}.

Kono, Daniel Y. (2007). ``Making Anarchy Work: International Legal
Institutions and Trade Cooperation''. In: \emph{The Journal of Politics}
69.3, pp.~746--759.
\url{https://doi.org/10.1111\%2Fj.1468-2508.2007.00572.x}.

Kono, Daniel Yuichi (2009). ``Market Structure, Electoral Institutions,
and Trade Policy''. In: \emph{International Studies Quarterly} 53.4,
pp.~885--906. \url{https://doi.org/10.1111\%2Fj.1468-2478.2009.00561.x}.

Li, Quan and David H. Sacko (2002). ``The (Ir)Relevance of Militarized
Interstate Disputes for International Trade''. In:
\emph{International Studies Quarterly} 46.1, pp.~11--43.
\url{https://doi.org/10.1111\%2F1468-2478.00221}.

Morrow, James D, Randolph M. Siverson, and Tressa E. Tabares (1998).
``The Political Determinants of International Trade: The Major Powers,
1907-1990''. In: \emph{American Political Science Review} 92.03,
pp.~649--661. \url{https://doi.org/10.2307\%2F2585487}.

Obinger, H, C. Schmitt, and R. Zohlnhofer (2013). ``Partisan Politics
and Privatization in Oecd Countries''. In:
\emph{Comparative Political Studies} 47.9, pp.~1294--1323.
\url{https://doi.org/10.1177\%2F0010414013495361}.

Pandya, Sonal S. (2010). ``Labor Markets and the Demand for Foreign
Direct Investment''. In: \emph{International Organization} 64.03,
pp.~389--409. \url{https://doi.org/10.1017\%2Fs0020818310000160}.

Souva, Mark, Dale L. Smith, and Shawn Rowan (2008). ``Promoting Trade:
The Importance of Market Protecting Institutions''. In:
\emph{The Journal of Politics} 70.2, pp.~383--392.
\url{https://doi.org/10.1017\%2Fs0022381608080377}.

Steinberg, Richard H (2002). ``In the Shadow of Law or Power?
Consensus-Based Bargaining and Outcomes in the Gatt/Wto''. In:
\emph{International Organization} 56.02, pp.~339-374

Walter, Stefanie (2010). ``Globalization and the Welfare State: Testing
the Microfoundations of the Compensation Hypothesis''. In:
\emph{International Studies Quarterly} 54.2, pp.~403--426.
\url{https://doi.org/10.1111\%2Fj.1468-2478.2010.00593.x}.

Webb, Michael C (1991). ``International Economic Structures, Government
Interests, and International Coordination of Macroeconomic Adjustment
Policies''. In: \emph{International Organization} 45.03, pp.~309-342

Wu, Wen-Chin (2014). ``When Do Dictators Decide to Open Trade
Regimes?-Inequality and Trade Openness in Authoritarian Countries''. In:
\emph{International Studies Quarterly}, pp.~n/a--n/a.
\url{https://doi.org/10.1111\%2Fisqu.12149}.

\subsection{Week 14, 11/14 - 11/18: International Political Economy:
International Finance and Foreign Aid \textbar{} 6
Readings}\label{week-14-1114---1118-international-political-economy-international-finance-and-foreign-aid-6-readings}

Ahmed, Faisal Z. (2012). ``The Perils of Unearned Foreign Income: Aid,
Remittances, and Government Survival''. In:
\emph{American Political Science Review} 106.01, pp.~146--165.
\url{https://doi.org/10.1017\%2Fs0003055411000475}.

Baccini, Leonardo and Johannes Urpelainen (2012). ``Strategic Side
Payments: Preferential Trading Agreements, Economic Reform, and Foreign
Aid''. In: \emph{The Journal of Politics} 74.4, pp.~932--949.
\url{https://doi.org/10.1017\%2Fs0022381612000485}.

Blanton, Shannon Lindsey and Robert G. Blanton (2007). ``What Attracts
Foreign Investors? an Examination of Human Rights and Foreign Direct
Investment''. In: \emph{The Journal of Politics} 69.1, pp.~143--155.
\url{https://doi.org/10.1111\%2Fj.1468-2508.2007.00500.x}.

Dreher, Axel, Jan-Egbert Sturm, and James Raymond Vreeland (2009).
``Development Aid and International Politics: Does Membership on the UN
Security Council Influence World Bank Decisions?'' In:
\emph{Journal of Development Economics} 88.1, pp.~1-18

Moon, Chungshik (2015). ``Foreign Direct Investment, Commitment
Institutions, and Time Horizon: How Some Autocrats Do Better Than
Others''. In: \emph{International Studies Quarterly} 59.2, pp.~344-356

Stone, Randall W. (2004). ``The Political Economy of Imf Lending in
Africa''. In: \emph{American Political Science Review} 98.04,
pp.~577--591. \url{https://doi.org/10.1017\%2Fs000305540404136x}.

\subsubsection{Supplementary \textbar{} 11
Readings:}\label{supplementary-11-readings}

Alesina, Alberto and David Dollar (2000). ``Who Gives Foreign Aid to
Whom and Why?'' In: \emph{Journal of economic growth} 5.1, pp.~33-63

Bernhard, William, J. Lawrence Broz, and William Roberts Clark (2002).
``The Political Economy of Monetary Institutions''. In:
\emph{International Organization} 56.4, pp.~693--723.
\url{https://doi.org/10.1162\%2F002081802760403748}.

Bodea, Cristina (2010). ``Exchange Rate Regimes and Independent Central
Banks: A Correlated Choice of Imperfectly Credible Institutions''. In:
\emph{International Organization} 64.03, pp.~411--442.
\url{https://doi.org/10.1017\%2Fs0020818310000111}.

Brooks, Sarah M. and Marcus J. Kurtz (2012). ``Paths to Financial Policy
Diffusion: Statist Legacies in Latin America's Globalization''. In:
\emph{International Organization} 66.01, pp.~95--128.
\url{https://doi.org/10.1017\%2Fs0020818311000385}.

Heinrich, Tobias (2013). ``When Is Foreign Aid Selfish, When Is It
Selfless?'' In: \emph{The Journal of Politics} 75.2, pp.~422--435.
\url{https://doi.org/10.1017\%2Fs002238161300011x}.

Nielsen, Richard a, Michael G Findley, Zachary S Davis, Tara Candland,
and Daniel L Nielson (2011). ``Foreign Aid Shocks as a Cause of Violent
Armed Conflict''. In: \emph{American Journal of Political Science} 55.2,
pp.~219-232

Quinn, Dennis P. and a. Maria Toyoda (2007). ``Ideology and Voter
Preferences as Determinants of Financial Globalization''. In:
\emph{American Journal of Political Science} 51.2, pp.~344--363.
\url{https://doi.org/10.1111\%2Fj.1540-5907.2007.00255.x}.

Savun, Burcu and Daniel C. Tirone (2012). ``Exogenous Shocks, Foreign
Aid, and Civil War''. In: \emph{International Organization} 66.03,
pp.~363--393. \url{https://doi.org/10.1017\%2Fs0020818312000136}.

Singer, David Andrew (2004). ``Capital Rules: The Domestic Politics of
International Regulatory Harmonization''. In:
\emph{International Organization} 58.03.
\url{https://doi.org/10.1017\%2Fs0020818304583042}.

Steinberg, David a. and Krishan Malhotra (2014). ``The Effect of
Authorita Rian Regime Type on Exchange Rate Policy''. In:
\emph{World Politics} 66.03, pp.~491--529.
\url{https://doi.org/10.1017\%2Fs0043887114000136}.

Stone, Randall W. (2008). ``The Scope of Imf Conditionality''. In:
\emph{International Organization} 62.04, p.~589.
\url{https://doi.org/10.1017\%2Fs0020818308080211}.

\subsection{Week 15, 11/21 - 11/25: International Political Economy:
Migration \textbar{} 6
Readings}\label{week-15-1121---1125-international-political-economy-migration-6-readings}

Bermeo, Sarah Blodgett and David Leblang (2015). ``Migration and Foreign
Aid''. In: \emph{International Organization} 69.03, pp.~627--657.
\url{https://doi.org/10.1017\%2Fs0020818315000119}.

Brady, David and Ryan Finnigan (2014). ``Does Immigration Undermine
Public Support for Social Policy?'' In:
\emph{American Sociological Review} 79.1, pp.~17--42.
\url{https://doi.org/10.1177\%2F0003122413513022}.

Dancygier, Rafaela M. and Michael J. Donnelly (2013). ``Sectoral
Economies, Economic Contexts, and Attitudes Toward Immigration''. In:
\emph{The Journal of Politics} 75.1, pp.~17--35.
\url{https://doi.org/10.1017\%2Fs0022381612000849}.

Hainmueller, Jens and Michael J. Hiscox (2010). ``Attitudes Toward
Highly Skilled and Low-Skilled Immigration: Evidence From a Survey
Experiment''. In: \emph{American Political Science Review} 104.01,
pp.~61--84. \url{https://doi.org/10.1017\%2Fs0003055409990372}.

Margalit, Yotam (2012). ``Lost in Globalization: International Economic
Integration and the Sources of Popular Discontent1''. In:
\emph{International Studies Quarterly} 56.3, pp.~484--500.
\url{https://doi.org/10.1111\%2Fj.1468-2478.2012.00747.x}.

Mirilovic, Nikola (2010). ``The Politics of Immigration: Dictatorship,
Development, and Defense''. In: \emph{Comparative Politics} 42.3,
pp.~273--292. \url{https://doi.org/10.5129\%2F001041510x12911363509675}.

\subsubsection{Supplementary \textbar{} 19
Readings:}\label{supplementary-19-readings}

Barry, Colin M, K Chad Clay, Michael E Flynn, and Gregory Robinson
(2014). ``Freedom of Foreign Movement, Economic Opportunities Abroad,
and Protest in Non-Democratic Regimes''. In:
\emph{Journal of Peace Research} 51.5, pp.~574--588.
\url{https://doi.org/10.1177\%2F0022343314537860}.

Bertocchi, Graziella and Chiara Strozzi (2010). ``The Evolution of
Citizenship: Economic and Institutional Determinants''. In:
\emph{The Journal of Law and Economics} 53.1, pp.~95--136.
\url{https://doi.org/10.1086\%2F600080}.

Bloom, Pazit Ben-Nun, Gizem Arikan, and Marie Courtemanche (2015).
``Religious Social Identity, Religious Belief, and Anti-Immigration
Sentiment''. In: \emph{American Political Science Review} 109.02,
pp.~203--221. \url{https://doi.org/10.1017\%2Fs0003055415000143}.

Brader, Ted, Nicholas a. Valentino, and Elizabeth Suhay (2008). ``What
Triggers Public Opposition to Immigration? Anxiety, Group Cues, and
Immigration Threat''. In: \emph{American Journal of Political Science}
52.4, pp.~959--978.
\url{https://doi.org/10.1111\%2Fj.1540-5907.2008.00353.x}.

Ehrlich, Sean D. and Eddie Hearn (2013). ``Does Compensating the Losers
Increase Support for Trade? an Experimental Test of the Embedded
Liberalism Thesis''. In: \emph{Foreign Policy Analysis} 10.2,
pp.~149--164. \url{https://doi.org/10.1111\%2Ffpa.12001}.

Esses, Victoria M, Kay Deaux, Richard N. Lalonde, and Rupert Brown
(2010). ``Psychological Perspectives on Immigration''. In:
\emph{Journal of Social Issues} 66.4, pp.~635--647.
\url{https://doi.org/10.1111\%2Fj.1540-4560.2010.01667.x}.

Freeman, G. P. (1986). ``Migration and the Political Economy of the
Welfare State''. In:
\emph{The ANNALS of the American Academy of Political and Social Science}
485.1, pp.~51--63. \url{https://doi.org/10.1177\%2F0002716286485001005}.

Green, Eva G. T. (2009). ``Who Can Enter? a Multilevel Analysis on
Public Support for Immigration Criteria Across 20 European Countries''.
In: \emph{Group Processes \& Intergroup Relations} 12.1, pp.~41--60.
\url{https://doi.org/10.1177\%2F1368430208098776}.

Hainmueller, Jens and Michael J. Hiscox (2007). ``Educated Preferences:
Explaining Attitudes Toward Immigration in Europe''. In:
\emph{International Organization} 61.02.
\url{https://doi.org/10.1017\%2Fs0020818307070142}.

Kibreab, Gaim (2006). ``Citizenship Rights and Repatriation of
Refugees''. In: \emph{International Migration Review} 37.1, pp.~24--73.
\url{https://doi.org/10.1111\%2Fj.1747-7379.2003.tb00129.x}.

Knoll, Benjamin R. (2009). ````And Who Is My Neighbor?`' Religion and
Immigration Policy Attitudes''. In:
\emph{Journal for the Scientific Study of Religion} 48.2, pp.~313--331.
\url{https://doi.org/10.1111\%2Fj.1468-5906.2009.01449.x}.

Leblang, David (2010). ``Familiarity Breeds Investment: Diaspora
Networks and International Investment''. In:
\emph{American Political Science Review} 104.03, pp.~584--600.
\url{https://doi.org/10.1017\%2Fs0003055410000201}.

Mayda, Anna Maria (2006). ``Who Is Against Immigration? a Cross-Country
Investigation of Individual Attitudes Toward Immigrants''. In:
\emph{Review of Economics and Statistics} 88.3, pp.~510--530.
\url{https://doi.org/10.1162\%2Frest.88.3.510}.

O'Rourke, Kevin H. and Richard Sinnott (2006). ``The Determinants of
Individual Attitudes Towards Immigration''. In:
\emph{European Journal of Political Economy} 22.4, pp.~838--861.
\url{https://doi.org/10.1016\%2Fj.ejpoleco.2005.10.005}.

Pérez, Efrén O. (2010). ``Explicit Evidence on the Import of Implicit
Attitudes: The Iat and Immigration Policy Judgments''. In:
\emph{Political Behavior} 32.4, pp.~517--545.
\url{https://doi.org/10.1007\%2Fs11109-010-9115-z}.

Rocha, Rene R, Benjamin R. Knoll, and Robert D. Wrinkle (2015).
``Immigration Enforcement and the Redistribution of Political Trust''.
In: \emph{The Journal of Politics} 77.4, pp.~901--913.
\url{https://doi.org/10.1086\%2F681810}.

Rudolph, Christopher (2003). ``Security and the Political Economy of
International Migration''. In: \emph{American Political Science Review}
97.04, pp.~603--620. \url{https://doi.org/10.1017\%2Fs000305540300090x}.

Singer, David Andrew (2010). ``Migrant Remittances and Exchange Rate
Regimes in the Developing World''. In:
\emph{American Political Science Review} 104.02, pp.~307--323.
\url{https://doi.org/10.1017\%2Fs0003055410000110}.

Sniderman, Paul M, Louk Hagendoorn, and Markus Prior (2004).
``Predisposing Factors and Situational Triggers: Exclusionary Reactions
to Immigrant Minorities''. In: \emph{American Political Science Review}
98.01, pp.~35--49. \url{https://doi.org/10.1017\%2Fs000305540400098x}.




\end{document}

\makeatletter
\def\@maketitle{%
  \newpage
%  \null
%  \vskip 2em%
%  \begin{center}%
  \let \footnote \thanks
    {\fontsize{18}{20}\selectfont\raggedright  \setlength{\parindent}{0pt} \@title \par}%
}
%\fi
\makeatother
